\documentclass[letter,12pt]{article}

\usepackage[T1]{fontenc}
\usepackage{lmodern}
\usepackage{textcomp}
\renewcommand*\familydefault{\sfdefault}

\usepackage[spanish]{babel}
\usepackage[utf8x]{inputenc}

\usepackage[pdftex]{graphicx}
\usepackage{pifont}
\usepackage[
pdfauthor={Carlos Eduardo Caballero Burgoa},%
pdftitle={macchiato},%
colorlinks,%
citecolor=black,%
filecolor=black,%
linkcolor=black,%
%urlcolor=black
pdftex]{hyperref}

\usepackage{fancyhdr}
\usepackage{lastpage}
\pagestyle{fancy}

% Para la primera página
\fancypagestyle{plain}{
\fancyhead[l]{}
\fancyhead[r]{}
\fancyhead[c]{}
\renewcommand{\headrulewidth}{0.5pt}
\fancyfoot[l]{SCESI \\ Sociedad Científica de Estudiantes de Sistemas e Informática}
\fancyfoot[c]{}
\fancyfoot[r]{\thepage/\pageref{LastPage}}
\renewcommand{\footrulewidth}{0.5pt}}

% Para el resto de páginas
\lhead{Proyecto Macchiato}
\chead{}
\renewcommand{\headrulewidth}{0.4pt}
\lfoot{SCESI \\ Sociedad Científica de Estudiantes de Sistemas e Informática \\ 
\url {http://www.scesi.org}}
\cfoot{}
\rfoot{\thepage/\pageref{LastPage}}
\renewcommand{\footrulewidth}{0.4pt}

\title{\bf Proyecto Macchiato}
\author{Carlos Eduardo Caballero Burgoa}

\begin{document}
\maketitle
\begin{center}\url {http://www.scesi.org}\end{center}
\pagebreak

\tableofcontents
\pagebreak

\section{Introducción}
Como parte de las construcciones de software orientado a soluciones primigenias, este documento
trata los asuntos referentes al desarrollo de un sistema para la reproducción de archivos de audio
en formato mp3 via online.

\section{Antecedentes}
En estos ultimos tiempos, la portabilidad se esta haciendo cada vez mas crucial a la hora de romper
brechas de accesibilidad a la información. Si bien los dispositivos mobiles, y los servicios en la nube
estan dominando tales mercados de consumo, no debe olvidarse que estos son en nuestro contexto
inalcanzables, lo que imperativamente requiere de una solución que pueda adaptarse a las cosas que
necesitamos y no tenemos.

Dicho en terminos simples, brindar acceso a recursos en linea, es la función prioritaria a tomar en
cuenta a la hora de poder reducir cualquier diferencia de accesibilidad a la información, y el
conocimiento entre las personas.

Aun asi, existe un problema emergente adicional, pues que cada dia que pasa, los recursos digitales
continuan siendo creados, haciendo del conocimiento algo muy dificil de rodear o englobar, por lo que
inevitablemente los metodos de adquisicion de conocimiento estan condenados a diversificarse.

\section{Definición del Problema}
Por lo mencionado anteriormente se define el problema como:

El creciente volumen de ficheros de audio, conlleva inexorablemente a una cada vez mayor inversión de
tiempo a la hora de escudriñar en los recursos de audio disponibles.

\section{Objetivo General}
Desarrollar un sistema web que permita a un usuario descargar, y reproducir en linea, ficheros en formato
mp3, de modo tal que pueda conocer y compartir sus preferencias musicales con otros usuarios.

\section{Objetivos Específicos}
\begin{itemize}
\item Crear un sistema de clasificación libre de ambiguiedades a la hora de ordenar taxonomicamente los
componentes de una colección de audio.
\item Proveer a esta clasificación una interfaz visual enriquecida.
\item Proveer al sistema de un reproductor en linea.
\item Desarrollar una funcionalidad para creación de listas de reproducción.
\end{itemize}

\section{Proceso de desarrollo}
A continuación se detallan los pasos a seguir:

\begin{enumerate}
\item Construcción de los modelos necesarios tal que estos puedan albergar la estructura de la
información que se tiene a disposición.
\item Creación de la base de datos que pueda albergar la información requerida.
\item Construcción de las interfaces necesarias para el uso del sistema.
\item Construcción de los controladores necesarios para la validación y filtrado de la información
entrante.
\item Configuración del servidor para el montado de directorios con archivos de audio.
\item Creación de las interfaces JSON para el intercambio de de información en el sistema (AJAX).
\item Poner el sitio construido a un servidor en producción (macchiato.scesi.org).
\end{enumerate}

\section{Herramientas}
En el axiomático caso de que a alguien le importe lo que estamos haciendo, se utilizaran herramientas
que faciliten el intercambio de código, e ideas que puedan mejorar la solución propuesta, entre otras
estas son:

\begin{enumerate}
\item FOS GNU/Linux: Como parte de la evaluación y a modo de contribución a este proyecto.
\item Apache2: Para el despliegue del sistema construido.
\item Git: Como herramienta de versionamiento y a futuro liberar el código del sistema en
el repositorio de proyectos github (https://github.com/).
\item NetBeans: Como IDE de desarrollo para PHP5.
\end{enumerate}

\section{Justificación}
Para todo aquel que vive de la portabilidad, tener la mayor cantidad de información en la nube, es
de vital importancia, ya sea como un respaldo de su propia información, o como una forma de fomentar el
libre intercambio de recursos entre usuarios.

Cabe mencionar que no solo hablamos de musica, sino de cualquier recurso de audio. Un claro ejemplo de
que estos tipos de recursos pueden contribuir al aprendizaje son los podcasts, y los audiolibros.

\end{document}
